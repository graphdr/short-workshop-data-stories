\documentclass[]{tufte-handout}

% ams
\usepackage{amssymb,amsmath}

\usepackage{ifxetex,ifluatex}
\usepackage{fixltx2e} % provides \textsubscript
\ifnum 0\ifxetex 1\fi\ifluatex 1\fi=0 % if pdftex
  \usepackage[T1]{fontenc}
  \usepackage[utf8]{inputenc}
\else % if luatex or xelatex
  \makeatletter
  \@ifpackageloaded{fontspec}{}{\usepackage{fontspec}}
  \makeatother
  \defaultfontfeatures{Ligatures=TeX,Scale=MatchLowercase}
  \makeatletter
  \@ifpackageloaded{soul}{
     \renewcommand\allcapsspacing[1]{{\addfontfeature{LetterSpace=15}#1}}
     \renewcommand\smallcapsspacing[1]{{\addfontfeature{LetterSpace=10}#1}}
   }{}
  \makeatother

\fi

% graphix
\usepackage{graphicx}
\setkeys{Gin}{width=\linewidth,totalheight=\textheight,keepaspectratio}

% booktabs
\usepackage{booktabs}

% url
\usepackage{url}

% hyperref
\usepackage{hyperref}

% units.
\usepackage{units}


\setcounter{secnumdepth}{-1}

% citations

\newlength{\cslhangindent}
\setlength{\cslhangindent}{1.5em}
% For Pandoc 2.8 to 2.11
\newenvironment{cslreferences}%
  {}%
  {\par}
% For pandoc 2.11+ using new --citeproc
\newlength{\csllabelwidth}
\setlength{\csllabelwidth}{3em}
\newlength{\cslentryspacingunit} % times entry-spacing
\setlength{\cslentryspacingunit}{\parskip}
\newenvironment{CSLReferences}[2] % #1 hanging-ident, #2 entry spacing
 {% don't indent paragraphs
  \setlength{\parindent}{0pt}
  % turn on hanging indent if param 1 is 1
  \ifodd #1
  \let\oldpar\par
  \def\par{\hangindent=\cslhangindent\oldpar}
  \fi
  % set entry spacing
  \setlength{\parskip}{#2\cslentryspacingunit}
 }%
 {}
\usepackage{calc}
\newcommand{\CSLBlock}[1]{#1\hfill\break}
\newcommand{\CSLLeftMargin}[1]{\parbox[t]{\csllabelwidth}{#1}}
\newcommand{\CSLRightInline}[1]{\parbox[t]{\linewidth - \csllabelwidth}{#1}}
\newcommand{\CSLIndent}[1]{\hspace{\cslhangindent}#1}

% pandoc syntax highlighting

% table with pandoc

% multiplecol
\usepackage{multicol}

% strikeout
\usepackage[normalem]{ulem}

% morefloats
\usepackage{morefloats}


% tightlist macro required by pandoc >= 1.14
\providecommand{\tightlist}{%
  \setlength{\itemsep}{0pt}\setlength{\parskip}{0pt}}

% title / author / date
\title[Graphical repertoire]{Expanding your graphical repertoire}
\author{Richard Layton}
\date{2026--02--10}


\begin{document}

\maketitle




\begin{marginfigure}
\emph{Richard Layton} resides online at

\begin{itemize}
\tightlist
\item
  \url{https://www.graphdoctor.com}\strut \\
\item
  \url{https://github.com/graphdr}
\end{itemize}
\end{marginfigure}

\includegraphics{chart-repertoire-handout-2026_files/figure-latex/unnamed-chunk-2-1}

\section{Variables, design, message}\label{variables-design-message}

\begin{marginfigure}
\emph{Trees, Maps, and Theorems} by Jean-luc Doumont (2009) inspired the
four main topics.
\end{marginfigure}

\includegraphics[width=0.7\linewidth,height=\textheight,keepaspectratio]{img/day2-cover-preview-3.png}

\includegraphics{chart-repertoire-handout-2026_files/figure-latex/unnamed-chunk-4-1}

\section{§ Showing evolution}\label{showing-evolution}

\begin{marginfigure}
Square brackets \emph{{[}i{]}} give the slide number.
\end{marginfigure}

\subsection{{[}4{]} Time series}\label{time-series}

\includegraphics[width=1.4\linewidth,height=\textheight,keepaspectratio]{../images/sdu-elec-by-month.png}

\section{§ Displaying distributions}\label{displaying-distributions}

\subsection{{[}6{]} Data}\label{data}

World speed skiing (km/hr) competitions 1953--1995

\begin{verbatim}
Key: <Year, Event, Sex>
     Year                 Event    Sex  Speed
    <int>                <fctr> <fctr>  <num>
 1:  1952        Speed Downhill   Male 167.85
 2:  1953        Speed Downhill   Male 168.86
 3:  1953             Speed One   Male 174.14
 4:  1957             Speed One   Male 177.47
 5:  1961        Speed Downhill   Male 165.42
 6:  1961             Speed One   Male 190.12
---                                          
86:  1993             Speed One   Male 170.30
87:  1994 Speed Downhill Junior Female 160.22
88:  1994 Speed Downhill Junior   Male 164.47
89:  1995 Speed Downhill Junior Female 166.52
90:  1995 Speed Downhill Junior Female 162.37
91:  1995 Speed Downhill Junior   Male 168.01
\end{verbatim}

\subsection{{[}7{]} Strip chart}\label{strip-chart}

\includegraphics[width=1.4\linewidth,height=\textheight,keepaspectratio]{../results/day-2-distributions-01.png}

\subsection{{[}8{]} Add a category}\label{add-a-category}

\includegraphics[width=1.4\linewidth,height=\textheight,keepaspectratio]{../results/day-2-distributions-02.png}

\subsection{{[}9{]} Add a second category}\label{add-a-second-category}

\includegraphics[width=1.4\linewidth,height=\textheight,keepaspectratio]{../results/day-2-distributions-03.png}

\subsection{{[}10{]} Data}\label{data-1}

MIDFIELD graduates (N = 270k), enrolled in Engineering, excluding 10th
and 90th quantiles

\begin{verbatim}
Key: <path, sex>
                  path    sex years_to_grad
                <char> <char>         <num>
     1: Nontraditional Female           3.9
     2: Nontraditional Female           1.9
     3: Nontraditional Female           3.9
     4: Nontraditional Female           5.3
     5: Nontraditional Female           5.1
    ---                                    
269053:    Traditional   Male           2.6
269054:    Traditional   Male           1.3
269055:    Traditional   Male           3.0
269056:    Traditional   Male           5.3
269057:    Traditional   Male           0.7
\end{verbatim}

\subsection{{[}11{]} Box and whisker chart}\label{box-and-whisker-chart}

\includegraphics[width=1.4\linewidth,height=\textheight,keepaspectratio]{../results/day-2-distributions-04.png}

\subsection{{[}12{]} Add a category}\label{add-a-category-1}

\includegraphics[width=1.4\linewidth,height=\textheight,keepaspectratio]{../results/day-2-distributions-05.png}

\subsection{{[}13{]} Combine a second
category}\label{combine-a-second-category}

\includegraphics[width=1.4\linewidth,height=\textheight,keepaspectratio]{../results/day-2-distributions-06.png}

\subsection{{[}14{]} Data}\label{data-2}

Revisiting the USD electrical data

\begin{verbatim}
Key: <date_peak, minutes>
      bill_year bill_month date_peak minutes    source    MW
          <int>     <fctr>    <char>   <num>    <char> <num>
   1:      2019        Jan 1/22/2019      15 Fuel cell  0.80
   2:      2019        Jan 1/22/2019      15     Meter  1.16
   3:      2019        Jan 1/22/2019      15     Solar  0.00
   4:      2019        Jan 1/22/2019      15     Total  1.96
   5:      2019        Jan 1/22/2019      30 Fuel cell  0.80
  ---                                                       
4604:      2018        Sep 9/13/2018    1425     Total  2.76
4605:      2018        Sep 9/13/2018    1440 Fuel cell  0.89
4606:      2018        Sep 9/13/2018    1440     Meter  1.75
4607:      2018        Sep 9/13/2018    1440     Solar  0.00
4608:      2018        Sep 9/13/2018    1440     Total  2.64
\end{verbatim}

\subsection{{[}15{]} Ignoring time}\label{ignoring-time}

\includegraphics[width=1.4\linewidth,height=\textheight,keepaspectratio]{../images/usd-energy-boxplot.png}

\subsection{{[}16{]} Discussion: Displaying
distributions}\label{discussion-displaying-distributions}

Quantitative test scores from a recent exam could be displayed as a
distribution. What categorical variable(s) could be added to create
comparative distributions?

\begin{marginfigure}
\pandocbounded{\includegraphics[keepaspectratio]{img/variables-cycle.png}}
\end{marginfigure}

\includegraphics{chart-repertoire-handout-2026_files/figure-latex/unnamed-chunk-11-1}

\section{§ Comparing data}\label{comparing-data}

\subsection{{[}18{]} Data}\label{data-3}

Representation at graduation in 3 engineering programs, 19 US
institutions, 1987--2018

\begin{verbatim}
          origin    sex Electrical Engr Computer Engr Computer Science
          <char> <char>           <int>         <int>            <int>
1: International Female            1865           140              365
2: International   Male            8530           993             1442
3:      Domestic Female           23426           702             2923
4:      Domestic   Male           90150          7481            13987
\end{verbatim}

\subsection{{[}19{]} Dot chart}\label{dot-chart}

\includegraphics[width=1.4\linewidth,height=\textheight,keepaspectratio]{../results/day-2-compare-01.png}

\subsection{{[}20{]} Add a second
category}\label{add-a-second-category-1}

\includegraphics[width=1.4\linewidth,height=\textheight,keepaspectratio]{../results/day-2-compare-02.png}

\subsection{{[}21{]} Exchange mapping of categorical
variables}\label{exchange-mapping-of-categorical-variables}

\includegraphics[width=1.4\linewidth,height=\textheight,keepaspectratio]{../results/day-2-compare-03.png}

\subsection{{[}22{]} Logarithmic scale for orders of magnitude
differences}\label{logarithmic-scale-for-orders-of-magnitude-differences}

\includegraphics[width=1.4\linewidth,height=\textheight,keepaspectratio]{../results/day-2-compare-04.png}

\subsection{{[}23{]} One program per facet}\label{one-program-per-facet}

\includegraphics[width=1.4\linewidth,height=\textheight,keepaspectratio]{../results/day-2-compare-05.png}

\subsection{{[}24{]} Add a third category}\label{add-a-third-category}

\includegraphics[width=1.4\linewidth,height=\textheight,keepaspectratio]{../results/day-2-compare-06.png}

\subsection{{[}25{]} Combine categories}\label{combine-categories}

\includegraphics[width=1.4\linewidth,height=\textheight,keepaspectratio]{../results/day-2-compare-07.png}

\subsection{{[}26{]} Discussion: Comparing
data}\label{discussion-comparing-data}

Consider Table 2 Campus Buildings in the USD Energy Master Plan
(p.~17-18). If we were to visualize these data in dot-chart form:

\begin{itemize}
\tightlist
\item
  Select the quantitative variable
\end{itemize}

\begin{marginfigure}
\pandocbounded{\includegraphics[keepaspectratio]{img/variables-cycle.png}}
\end{marginfigure}

\includegraphics{chart-repertoire-handout-2026_files/figure-latex/unnamed-chunk-14-1}

\begin{itemize}
\tightlist
\item
  Select a categorical variable for the rows
\end{itemize}

\includegraphics{chart-repertoire-handout-2026_files/figure-latex/unnamed-chunk-15-1}

\begin{itemize}
\tightlist
\item
  How would you order the rows?
\end{itemize}

\includegraphics{chart-repertoire-handout-2026_files/figure-latex/unnamed-chunk-16-1}

\begin{itemize}
\tightlist
\item
  Select a second categorical variable for the facets
\end{itemize}

\section{§ Revealing correlations}\label{revealing-correlations}

\subsection{{[}28{]} Data}\label{data-4}

Engineering students at 14 institutions persisting to year 4 and
graduating by year 6, 1987-2019

\begin{verbatim}
    institution    sex    y4    y6
         <char> <char> <int> <int>
 1:           A Female  4953  4525
 2:           A   Male 17897 16312
 3:           B Female  2834  3316
---                               
26:           N   Male  1338   838
27:           P Female   457   283
28:           P   Male   827   447
\end{verbatim}

\subsection{{[}29{]} Scatterplots are designed to reveal
correlation}\label{scatterplots-are-designed-to-reveal-correlation}

\includegraphics[width=0.51\linewidth,height=\textheight,keepaspectratio]{../results/day-2-correlation-01.png}

\subsection{{[}30{]} Add a category}\label{add-a-category-2}

\includegraphics[width=0.51\linewidth,height=\textheight,keepaspectratio]{../results/day-2-correlation-02.png}

\subsection{{[}31{]} One facet per sex}\label{one-facet-per-sex}

\includegraphics[width=1.2\linewidth,height=\textheight,keepaspectratio]{../results/day-2-correlation-03.png}

\includegraphics{chart-repertoire-handout-2026_files/figure-latex/unnamed-chunk-18-1}

\subsection{{[}32{]} One facet per
institution}\label{one-facet-per-institution}

\includegraphics[width=1.5\linewidth,height=\textheight,keepaspectratio]{../results/day-2-correlation-04.png}

\includegraphics{chart-repertoire-handout-2026_files/figure-latex/unnamed-chunk-19-1}

\subsection{{[}33{]} Change the quantitative
variable}\label{change-the-quantitative-variable}

Engineering students at 14 institutions persisting to year 4 and
graduating by year 6, 1987--2019

\includegraphics{chart-repertoire-handout-2026_files/figure-latex/unnamed-chunk-20-1}

\includegraphics[width=1.3\linewidth,height=\textheight,keepaspectratio]{../results/day-2-correlation-05.png}

\includegraphics{chart-repertoire-handout-2026_files/figure-latex/unnamed-chunk-21-1}

\subsection{{[}34{]} Discussion: Revealing
correlations}\label{discussion-revealing-correlations}

Figure 17 (p.~36) of the USD Energy Master Plan has the form of a
scatterplot. We must assume that the authors are attempting to discover
a correlation between two quantitative variables.

\begin{marginfigure}
\pandocbounded{\includegraphics[keepaspectratio]{img/variables-cycle.png}}
\end{marginfigure}

\begin{itemize}
\tightlist
\item
  What are the two quantitative variables?
\end{itemize}

\includegraphics{chart-repertoire-handout-2026_files/figure-latex/unnamed-chunk-23-1}

\begin{itemize}
\tightlist
\item
  Do the variables appear to be correlated?
\end{itemize}

\includegraphics{chart-repertoire-handout-2026_files/figure-latex/unnamed-chunk-24-1}

\begin{itemize}
\tightlist
\item
  Is the linear curve fit justified?
\end{itemize}

\section{§ Closing discussion}\label{closing-discussion}

\begin{marginfigure}
\pandocbounded{\includegraphics[keepaspectratio]{img/variables-cycle.png}}
\end{marginfigure}

\subsection{{[}36{]} Variables, design,
message}\label{variables-design-message-1}

\begin{itemize}
\item
  Chart design depends on your variables
\item
  Chart design depends on the message the data convey
\item
  Continue to expand your repertoire of chart types
\end{itemize}

\includegraphics{chart-repertoire-handout-2026_files/figure-latex/unnamed-chunk-26-1}

\section*{References}\label{references}
\addcontentsline{toc}{section}{References}

\phantomsection\label{refs}
\begin{CSLReferences}{1}{0}
\bibitem[\citeproctext]{ref-Clarke+etal:2021}
Clarke, Steven, Michael Anderson, Daniela Aramayo, Dominic Molinari,
Arthur Tseng, Zoe Warp, and John Ko. 2021. {``University of San Diego
Energy Master Plan.''} Anaheim, CA: Willdan Energy Solutions.

\bibitem[\citeproctext]{ref-Doumont:2009}
Doumont, Jean-luc. 2009. \emph{{Trees, Maps, and Theorems}}. Belgium:
{Principiae}.

\bibitem[\citeproctext]{ref-R-GDAdata:2015}
Unwin, Antony. 2015. \emph{GDAdata: Datasets for the Book Graphical Data
Analysis with r}. \url{https://CRAN.R-project.org/package=GDAdata}.

\end{CSLReferences}



\end{document}
