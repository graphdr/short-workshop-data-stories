\documentclass[]{tufte-handout}

% ams
\usepackage{amssymb,amsmath}

\usepackage{ifxetex,ifluatex}
\usepackage{fixltx2e} % provides \textsubscript
\ifnum 0\ifxetex 1\fi\ifluatex 1\fi=0 % if pdftex
  \usepackage[T1]{fontenc}
  \usepackage[utf8]{inputenc}
\else % if luatex or xelatex
  \makeatletter
  \@ifpackageloaded{fontspec}{}{\usepackage{fontspec}}
  \makeatother
  \defaultfontfeatures{Ligatures=TeX,Scale=MatchLowercase}
  \makeatletter
  \@ifpackageloaded{soul}{
     \renewcommand\allcapsspacing[1]{{\addfontfeature{LetterSpace=15}#1}}
     \renewcommand\smallcapsspacing[1]{{\addfontfeature{LetterSpace=10}#1}}
   }{}
  \makeatother

\fi

% graphix
\usepackage{graphicx}
\setkeys{Gin}{width=\linewidth,totalheight=\textheight,keepaspectratio}

% booktabs
\usepackage{booktabs}

% url
\usepackage{url}

% hyperref
\usepackage{hyperref}

% units.
\usepackage{units}


\setcounter{secnumdepth}{-1}

% citations

\newlength{\cslhangindent}
\setlength{\cslhangindent}{1.5em}
% For Pandoc 2.8 to 2.11
\newenvironment{cslreferences}%
  {}%
  {\par}
% For pandoc 2.11+ using new --citeproc
\newlength{\csllabelwidth}
\setlength{\csllabelwidth}{3em}
\newlength{\cslentryspacingunit} % times entry-spacing
\setlength{\cslentryspacingunit}{\parskip}
\newenvironment{CSLReferences}[2] % #1 hanging-ident, #2 entry spacing
 {% don't indent paragraphs
  \setlength{\parindent}{0pt}
  % turn on hanging indent if param 1 is 1
  \ifodd #1
  \let\oldpar\par
  \def\par{\hangindent=\cslhangindent\oldpar}
  \fi
  % set entry spacing
  \setlength{\parskip}{#2\cslentryspacingunit}
 }%
 {}
\usepackage{calc}
\newcommand{\CSLBlock}[1]{#1\hfill\break}
\newcommand{\CSLLeftMargin}[1]{\parbox[t]{\csllabelwidth}{#1}}
\newcommand{\CSLRightInline}[1]{\parbox[t]{\linewidth - \csllabelwidth}{#1}}
\newcommand{\CSLIndent}[1]{\hspace{\cslhangindent}#1}

% pandoc syntax highlighting

% table with pandoc
\usepackage{longtable,booktabs,array}
\usepackage{calc} % for calculating minipage widths
% Correct order of tables after \paragraph or \subparagraph
\usepackage{etoolbox}
\makeatletter
\patchcmd\longtable{\par}{\if@noskipsec\mbox{}\fi\par}{}{}
\makeatother
% Allow footnotes in longtable head/foot
\IfFileExists{footnotehyper.sty}{\usepackage{footnotehyper}}{\usepackage{footnote}}
\makesavenoteenv{longtable}

% multiplecol
\usepackage{multicol}

% strikeout
\usepackage[normalem]{ulem}

% morefloats
\usepackage{morefloats}


% tightlist macro required by pandoc >= 1.14
\providecommand{\tightlist}{%
  \setlength{\itemsep}{0pt}\setlength{\parskip}{0pt}}

% title / author / date
\title[More effective charts]{Creating more effective charts}
\author{Richard Layton}
\date{2026--02--10}


\begin{document}

\maketitle




\begin{marginfigure}
\emph{Richard Layton} resides online at

\begin{itemize}
\tightlist
\item
  \url{https://www.graphdoctor.com}\strut \\
\item
  \url{https://github.com/graphdr}
\end{itemize}
\end{marginfigure}

\section{Perception, reasoning, and
credibility}\label{perception-reasoning-and-credibility}

\begin{marginfigure}
\emph{Creating More Effective Graphs} by Naomi Robbins (2013) inspired
the session title and Chapter 2, ``Limitations of some common graphs,''
inspired our exercises.
\end{marginfigure}

\includegraphics[width=0.75\linewidth,height=\textheight,keepaspectratio]{img/handout-day1-outline-2024.png}

\includegraphics{effective-charts-handout-2026_files/figure-latex/unnamed-chunk-3-1}

\section{§ Effective alternatives to pie
charts}\label{effective-alternatives-to-pie-charts}

\subsection{Judging pie slices is a low-accuracy
task}\label{judging-pie-slices-is-a-low-accuracy-task}

\includegraphics{effective-charts-handout-2026_files/figure-latex/unnamed-chunk-4-1}

\includegraphics[width=1.5\linewidth,height=\textheight,keepaspectratio]{img/handout-01-01-v2.png}

\includegraphics{effective-charts-handout-2026_files/figure-latex/unnamed-chunk-5-1}

\begin{marginfigure}
Data source: World Bank (2022)
\end{marginfigure}

\subsection{Judging values along a common axis is a high-accuracy
task}\label{judging-values-along-a-common-axis-is-a-high-accuracy-task}

\begin{marginfigure}
\includegraphics[width=0.5\linewidth,height=\textheight,keepaspectratio]{img/pie-preview.png}
\end{marginfigure}
\marginnote{The data from the pie chart is shown below as dots along a common scale.}

\begin{itemize}
\item
  The new chart displays the same data
\item
  \emph{Visually estimate} the percentages using the new chart
\item
  Fill-in the blanks in the table
\end{itemize}

\includegraphics{effective-charts-handout-2026_files/figure-latex/unnamed-chunk-8-1}

\includegraphics[width=1.5\linewidth,height=\textheight,keepaspectratio]{img/handout-01-02.png}

\includegraphics{effective-charts-handout-2026_files/figure-latex/unnamed-chunk-9-1}

\subsection{3D effects distort our judgment even
further}\label{d-effects-distort-our-judgment-even-further}

\includegraphics{effective-charts-handout-2026_files/figure-latex/unnamed-chunk-10-1}

\includegraphics[width=1.5\linewidth,height=\textheight,keepaspectratio]{img/handout-01-03-v2.png}

\includegraphics{effective-charts-handout-2026_files/figure-latex/unnamed-chunk-11-1}

\begin{marginfigure}
Data source: World Bank (2022)
\end{marginfigure}

\subsection{Again, a common scale improves our visual
judgments}\label{again-a-common-scale-improves-our-visual-judgments}

\begin{marginfigure}
\includegraphics[width=0.5\linewidth,height=\textheight,keepaspectratio]{img/three-d-pie-preview.png}
\end{marginfigure}
\marginnote{The data from the pie chart is shown below as dots along a common scale.}

\begin{itemize}
\item
  The new chart displays the same data
\item
  \emph{Visually estimate} the percentages using the new chart
\item
  Fill-in the blanks in the table
\end{itemize}

\includegraphics{effective-charts-handout-2026_files/figure-latex/unnamed-chunk-14-1}

\includegraphics[width=1.5\linewidth,height=\textheight,keepaspectratio]{img/handout-01-04.png}

\includegraphics{effective-charts-handout-2026_files/figure-latex/unnamed-chunk-15-1}

\section{§ Effective alternatives to bar
charts}\label{effective-alternatives-to-bar-charts}

\subsection{3D effects always distort our
judgment}\label{d-effects-always-distort-our-judgment}

\includegraphics{effective-charts-handout-2026_files/figure-latex/unnamed-chunk-16-1}

\includegraphics[width=1.5\linewidth,height=\textheight,keepaspectratio]{img/handout-01-05-v2.png}

\includegraphics{effective-charts-handout-2026_files/figure-latex/unnamed-chunk-17-1}

\begin{marginfigure}
Data source: World Bank (2022)
\end{marginfigure}

\subsection{Same data---without 3D effects---along a common
scale}\label{same-datawithout-3d-effectsalong-a-common-scale}

\begin{marginfigure}
\includegraphics[width=0.5\linewidth,height=\textheight,keepaspectratio]{img/three-d-bar-preview.png}
\end{marginfigure}
\marginnote{The data from the 3D bar chart is shown below as dots along a common scale.}

\begin{itemize}
\item
  The new chart displays the same data
\item
  Visually estimate the percentages using the new chart
\item
  Fill-in the blanks in the table
\end{itemize}

\includegraphics[width=1.5\linewidth,height=\textheight,keepaspectratio]{img/handout-01-06.png}

\includegraphics{effective-charts-handout-2026_files/figure-latex/unnamed-chunk-20-1}

\subsection{With a zero baseline and no 3D effects, bars are
OK}\label{with-a-zero-baseline-and-no-3d-effects-bars-are-ok}

\begin{itemize}
\item
  Zero baseline avoids deception
\item
  Ordered by data values
\item
  Only the endpoint encodes information
\end{itemize}

\subsection{Consider dot charts for}\label{consider-dot-charts-for}

\begin{itemize}
\item
  Visually comparing quantities
\item
  Replacing most pie and bar charts
\end{itemize}

\begin{marginfigure}
Default bar chart:
\pandocbounded{\includegraphics[keepaspectratio]{img/handout-01-08.png}}
Ordered by magnitude:
\pandocbounded{\includegraphics[keepaspectratio]{img/handout-01-09.png}}
Omitting the fill color:
\pandocbounded{\includegraphics[keepaspectratio]{img/handout-01-10.png}}
Produces a dot chart:
\pandocbounded{\includegraphics[keepaspectratio]{img/handout-01-11.png}}
\end{marginfigure}

\subsection{Notes}\label{notes}

\includegraphics{effective-charts-handout-2026_files/figure-latex/unnamed-chunk-22-1}

\section{§ Aligning the design to the
story}\label{aligning-the-design-to-the-story}

Redesigning a chart to find what stories might be in the data

\newthought{The appendix of the USD energy report} includes this
stacked-column chart.

\includegraphics{effective-charts-handout-2026_files/figure-latex/unnamed-chunk-23-1}

\includegraphics[width=1.55\linewidth,height=\textheight,keepaspectratio]{img/sdu-energy-00.png}

\newthought{After we discuss these data,} write your thoughts in
response to these prompts:

\begin{itemize}
\tightlist
\item
  What does the chart show clearly?
\end{itemize}

\includegraphics{effective-charts-handout-2026_files/figure-latex/unnamed-chunk-24-1}

\begin{itemize}
\tightlist
\item
  What does the chart not show clearly?
\end{itemize}

\includegraphics{effective-charts-handout-2026_files/figure-latex/unnamed-chunk-25-1}

\begin{itemize}
\tightlist
\item
  Describe the problems faced by a reader trying to compare and contrast
  trends over 12 months (12 charts like this one).
\end{itemize}

\subsection{Redesign}\label{redesign}

In this multi-faceted chart, we can see the trends for each of the three
demand sources independently of one another, plus the total in the top
row, spanning the full year.

\includegraphics{effective-charts-handout-2026_files/figure-latex/unnamed-chunk-26-1}

\includegraphics[width=1.55\linewidth,height=\textheight,keepaspectratio]{img/sdu-elec-by-month.png}

\newthought{After we discuss these data,} write your thoughts in
response to these prompts:

\begin{itemize}
\tightlist
\item
  What observations (even obvious ones) can you make?
\end{itemize}

\includegraphics{effective-charts-handout-2026_files/figure-latex/unnamed-chunk-27-1}

\begin{itemize}
\tightlist
\item
  What attributes of this design, compared to the original stacked-bar
  chart, \emph{help the reader} to find stories in the data?
\end{itemize}

\includegraphics{effective-charts-handout-2026_files/figure-latex/unnamed-chunk-28-1}

\subsection{A second redesign}\label{a-second-redesign}

Electrical power is recorded on the peak demand day in each month. The
three measured quantities (solar, fuel cell, and metered kW) are
dependent on two discrete time variables.

\begin{description}
\item[month]
of the fiscal year, from July 2018 to June 2019
\item[time of day]
in 15-minute discrete intervals from 12:15 am to the following midnight
(or, using a 24 hour clock, from 0015 hours to 2400 hours).
\end{description}

In the previous design, we used time of day for the horizontal scale of
each panel and month to organize the sequence of panels.

\includegraphics{effective-charts-handout-2026_files/figure-latex/unnamed-chunk-29-1}

\includegraphics[width=1.55\linewidth,height=\textheight,keepaspectratio]{img/sdu-total-by-month.png}

\includegraphics{effective-charts-handout-2026_files/figure-latex/unnamed-chunk-30-1}

Conceptually, a discrete sequence of 15-minute intervals over a day and
a discrete sequence of months over a year a quite similar---just
different intervals over different spans.

Thus, we can switch the graphical roles of the two discrete time
variables. We can use fiscal-year months for the horizontal scale of
each panel and time of day to organize the sequence of panels.

\includegraphics{effective-charts-handout-2026_files/figure-latex/unnamed-chunk-31-1}

\includegraphics[width=0.4\linewidth,height=\textheight,keepaspectratio]{img/sdu-hour-1-panel.png}

\includegraphics{effective-charts-handout-2026_files/figure-latex/unnamed-chunk-32-1}

\includegraphics[width=1.55\linewidth,height=\textheight,keepaspectratio]{img/sdu-elec-by-hour.png}

\newthought{After we discuss these data,} write your thoughts in
response to these prompts:

\begin{itemize}
\tightlist
\item
  What observations can you make?
\end{itemize}

\includegraphics{effective-charts-handout-2026_files/figure-latex/unnamed-chunk-33-1}

\begin{itemize}
\tightlist
\item
  Both of the new charts (panels ordered by month and panels ordered by
  hour) are certainly more compact than the 12 original charts in the
  report appendix. Why do you think the authors presented the charts
  they did?
\end{itemize}

\includegraphics{effective-charts-handout-2026_files/figure-latex/unnamed-chunk-34-1}

\section{§ Advice from experts}\label{advice-from-experts}

Match the expert to the advice.

\newthought{Fill in the blanks} with letters A--D.

\begin{longtable}[]{@{}lll@{}}
\toprule\noalign{}
Expert & Letter & Emphasizes the importance of \\
\midrule\noalign{}
\endhead
\bottomrule\noalign{}
\endlastfoot
& & \\
A. Alberto Cairo & \_\_\_\_\_\_ & message \\
& & \\
B. Jean-luc Doumont & \_\_\_\_\_\_ & variables \\
& & \\
C. Stephanie Evergreen & \_\_\_\_\_\_ & revealing the complex \\
& & \\
D. Edward Tufte & \_\_\_\_\_\_ & knowing your main point \\
& & \\
& \_\_\_\_\_\_ & not lying to yourself \\
& & \\
\end{longtable}

\includegraphics{effective-charts-handout-2026_files/figure-latex/unnamed-chunk-36-1}

\section{Ideas to consider}\label{ideas-to-consider}

\begin{itemize}
\tightlist
\item
  Characterize the data structure and content
\end{itemize}

\includegraphics{effective-charts-handout-2026_files/figure-latex/unnamed-chunk-37-1}

\begin{itemize}
\tightlist
\item
  Explore a story's context, causality, and complexity
\end{itemize}

\includegraphics{effective-charts-handout-2026_files/figure-latex/unnamed-chunk-38-1}

\begin{itemize}
\tightlist
\item
  Align visual and verbal logic by revising iteratively
\end{itemize}

\includegraphics{effective-charts-handout-2026_files/figure-latex/unnamed-chunk-39-1}

\begin{itemize}
\tightlist
\item
  Edit to suit the rhetorical goals for each audience
\end{itemize}

\includegraphics{effective-charts-handout-2026_files/figure-latex/unnamed-chunk-40-1}

\begin{itemize}
\tightlist
\item
  Control every pixel---avoid thoughtless conformity
\end{itemize}

\includegraphics{effective-charts-handout-2026_files/figure-latex/unnamed-chunk-41-1}

\begin{itemize}
\tightlist
\item
  Question are you seeing only what you want to believe?
\end{itemize}

\section*{References}\label{references}
\addcontentsline{toc}{section}{References}

\phantomsection\label{refs}
\begin{CSLReferences}{1}{0}
\bibitem[\citeproctext]{ref-Cairo:2019}
Cairo, Alberto. 2019. \emph{{How Charts Lie}}. New York: {W.W. Norton}.

\bibitem[\citeproctext]{ref-Clarke+etal:2021}
Clarke, Steven, Michael Anderson, Daniela Aramayo, Dominic Molinari,
Arthur Tseng, Zoe Warp, and John Ko. 2021. {``University of San Diego
Energy Master Plan.''} Anaheim, CA: Willdan Energy Solutions.

\bibitem[\citeproctext]{ref-Doumont:2009}
Doumont, Jean-luc. 2009. \emph{{Trees, Maps, and Theorems}}. Belgium:
{Principiae}.

\bibitem[\citeproctext]{ref-Evergreen:2017}
Evergreen, Stephanie D. H. 2017. \emph{{Effective Data Visualization}}.
Thousand Oaks, CA: {Sage}.

\bibitem[\citeproctext]{ref-Robbins:2013}
Robbins, Naomi. 2013. \emph{{Creating More Effective Graphs}}. Wayne,
NJ: {Chart House}.

\bibitem[\citeproctext]{ref-Tufte:1983}
Tufte, Edward. 1983. \emph{{The Visual Display of Quantitative
Information}}. Cheshire, CT: {Graphics Press}.

\bibitem[\citeproctext]{ref-WorldBank:2022}
World Bank. 2022-01. {``{Population total for United States}.''}
{Federal Reserve Bank of St. Louis}.
\url{https://fred.stlouisfed.org/series/POPTOTUSA647NWDB}.

\end{CSLReferences}



\end{document}
